\documentclass{article}
\usepackage[utf8]{inputenc}
\usepackage{graphicx}
\graphicspath{{figures/}}  % then \includegraphics{createmode_0.jpg}
% for color box
\usepackage{tcolorbox}
\tcbuselibrary{listings, skins, breakable}
% math
\usepackage{amsmath, amssymb, amsthm}
% for float formatting options
\usepackage{float}
% tikz
\usepackage{tikz}
\usetikzlibrary{arrows.meta, positioning, shapes.geometric, fit}
% for tikz formatting
\usepackage{adjustbox}
% for links
\usepackage[colorlinks=true, linkcolor=blue, urlcolor=magenta,
citecolor=violet]{hyperref}

\title{Vulkan Notes}
\author{Joseph Grimsic}
\date{November 2025}
\usepackage[letterpaper, top=1.3in, bottom=1.3in, left=1.3in,
right=1.3in, heightrounded]{geometry}
\renewcommand{\baselinestretch}{1.15} % Line spacing
\setlength{\parindent}{0pt}
\setlength{\parskip}{0.8em}
\begin{document}
\maketitle

% --- This is the magic command ---
\tableofcontents

\newpage % Starts the main content on a new page

\section*{Vulkan Questions}
\addcontentsline{toc}{section}{Vulkan Questions}
\begin{itemize}
  \item what are the things that specifically separates OpenGL and Vulkan?
  \item What is sType?
  \item What is a command pool?
  \item How can we abstract commands/ should we?
  \item Do we need to work with commands regularly to parallelize our
    game engine?
\end{itemize}

\section*{What If I Use Sascha as a starting point}
\addcontentsline{toc}{section}{What If I Use Sascha as a starting point}
\begin{itemize}
  \item How would I work with or change this program to write a game engine?
  \item What abstractions would I want to make?
  \item What if I am a solo dev—does that change how I should abstract things?
  \item How can I black box the lesser parts of Vulkan I don't need
    to know that deeply?
  \item What levels of abstraction will I usually be using?
\end{itemize}

\section*{Vulkan Resources}
\addcontentsline{toc}{section}{Vulkan Resources}
\begin{itemize}
  \item
    \href{https://github.com/SaschaWillems/Vulkan/blob/master/examples/triangle/triangle.cpp}{Vulkan
    Triangle Example}
  \item \href{https://github.com/KhronosGroup/Vulkan-Hpp}{Vulkan Helper?}

  \item
    \href{https://www.reddit.com/r/GraphicsProgramming/comments/tc9d2u/the_current_state_of_gpu_apis_and_why_i_wish_vez/}{Interesting
    opinions on Vulkan}

  \item
    \href{https://docs.vulkan.org/tutorial/latest/00_Introduction.html}{Vulkan
    Tutorial By Khronos Group}
  \item \href{https://vkguide.dev/}{vkguide.dev}
\end{itemize}

\section*{Sascha Willems Helpers}
\addcontentsline{toc}{section}{Sascha Willems Helpers}
\subsection*{\_CreateInfo and \_Info}
\addcontentsline{toc}{subsection}{\_CreateInfo and \_Info}
I don't know what Sascha's functions are entirely for\textemdash
except that they attempt to reduce Vulkan boilerplate. However, it
seems that every function returns two kinds of pointers:
\begin{verbatim}
*_CreateInfo
*_Info
\end{verbatim}
\par We are using some kind of struct\textemdash but I'm not sure what the
struct is for. The struct probably holds whatever these two kinds of
pointers point to?

\par So actually, the \* in \*\_CreateInfo and
\*\_Info represent the wildcard operator, and is just to show
the naming convention in the documentation. These names refer to
structs that have state information for Vulkan.

\subsection*{Commands}
\addcontentsline{toc}{subsection}{Commands}
\par Commands are the main part of Vulkan. Commands are how we talk
directly to the GPU.

\par Commands are normally stored in a \textbf{command buffer}, so we
can batch them together. We usually call the act of intializing a
command buffer \textbf{recording} the command buffer. In the Vulkan
workflow, we will have record a command buffer using a
\textbf{command pool}. We will have one command pool for every
\textbf{queue family}.

\section*{Sascha Repository Notes}
\addcontentsline{toc}{section}{Sascha Repository Notes}
\par Git does not automatically download submodules

\par To update submodules:
\begin{verbatim}
git submodule update --init --recursive
\end{verbatim}

\par The following is an example clean build script specific to Wayland:

\begin{verbatim}
#!/usr/bin/env bash
set -e

BUILD_DIR="build"

echo ">>> Removing old build directory..."
rm -rf "$BUILD_DIR"

echo ">>> Creating new build directory..."
mkdir "$BUILD_DIR"
cd "$BUILD_DIR"

echo ">>> Running CMake configure..."
cmake .. \
  -DCMAKE_EXPORT_COMPILE_COMMANDS=ON \
  -DUSE_WAYLAND_WSI=OFF

echo ">>> Building..."
cmake --build . -- -j"$(nproc)"

# Optionally copy compile_commands.json to project root
if [ -f compile_commands.json ]; then
    echo ">>> Copying compile_commands.json to project root..."
    cp compile_commands.json ..
fi

echo ">>> Done."
\end{verbatim}

Then you can run the binaries, for example:
\begin{verbatim}
./build/bin/bloom
\end{verbatim}

\section*{Reading Through The Vulkan Tutorial}
\addcontentsline{toc}{section}{Reading Through The Vulkan Tutorial}
Here is a high level of how I would currently describe the Vulkan setup:

\begin{enumerate}
  \item create instance
  \item device selection
  \item create window
  \item then some other stuff
\end{enumerate}

here is my second attempt at that
\begin{enumerate}
  \item create instance + device selection
  \item queue families stuff
  \item create window + swapchain
  \item framebuffer stuff
  \item probably something else
\end{enumerate}

\par we are getting closer, but honestly maybe it's not that important. i
am considering reading throught the vulkan tutorial but i am unsure
as when to just jump in. i could probably save some time in the
future by reading a little bit now but i suspect i will not retain
that much information. i guess i will take some notes.
\par first i will skip to the hello triangle in their example and think of
some questions i have..
\par some things that stand out to me:
\begin{itemize}
  \item item (will i finish?)
\end{itemize}

\section*{Going Through Sascha's Repository}
\addcontentsline{toc}{section}{Going Through Sascha's Repository}
\subsection*{Idea}
\addcontentsline{toc}{subsection}{Idea}
I am thinking this will be faster to learn as Sascha provides a
standard hello triangle program (meant to be as vanilla as possible)
and shows his evolution (you can see comments in his READ.md) of
abstractions. This will likely make the most sense to study because
we know that Sascha has a
\href{https://github.com/SaschaWillems/Vulkan?tab=readme-ov-file}{very
popular repo} so we can assume others think it is good and that it is
a valid idea for me to pursue.

\subsection*{Notes On Sascha}
\addcontentsline{toc}{subsection}{Notes On Sascha}
\par Let's start with is Sascha's claimed vanilla hello triangle similar
to the one in the khronos vulkan tutorial?

\subsection*{Questions From Sascha's Repo:}
\addcontentsline{toc}{subsection}{Questions From Sascha's Repo:}
\begin{itemize}
    % \item what is a descriptor set in vulkan?
  \item what is the graphics queue?
    % \item what is a vk semaphore?
    % \item what is vk device?
    % \item what is vk null handle?
    % \item what is vk descriptor set?
  \item what is vk pipeline layout?
  \item how to read/write to a glm::mat4?
    % \item what is vk descriptor set layout?
  \item how do we interact with vk command pool?
  \item what does this do?
  \begin{verbatim}
  VulkanExample() : VulkanExampleBase() {
  \end{verbatim}

\end{itemize}

\section*{Theory}
\addcontentsline{toc}{section}{Theory}
\subsection*{Descriptor Set}
\addcontentsline{toc}{subsection}{Descriptor Set}
\par Descriptor sets are sets of pointers to resources that shaders
want access to. In OpenGL, you might 'bind' one buffer at a time.
Vulkan expects you to bind a descriptor set all together.
\par Similar to \textbf{command pools} there are \textbf{descriptor pools}.

\subsection*{Descriptor Pools}
\addcontentsline{toc}{subsection}{Descriptor Pools}
\par To allocate a descriptor set, we first intialize a descriptor
pool. When setting up a descriptor pool, we tell vulkan there will be
'x' number of sets with 'y' number of descriptors

\subsection*{Fences}
\addcontentsline{toc}{subsection}{Fences}
\par A fence is used to sync the CPU and GPU. I don't know how it
does this. A fence can be \textbf{signaled} by the GPU, so that the
CPU knows that the GPU has finished drawing the frame.

\subsection*{Semaphore}
\addcontentsline{toc}{subsection}{Semaphore}
\par Similarly, a \textbf{semaphore} is used to sync different parts
of the GPU. Some examples of semaphores are:
\begin{itemize}
  \item \textbf{renderSemaphore}
  \item \textbf{swapchainSemaphore}
\end{itemize}
\par There is also a \textbf{complete semaphore}, but I don't know
what that is for yet.

\subsection*{Projection Matrix}
\addcontentsline{toc}{subsection}{Projection Matrix}
\par This has something to do with how objects look when they are
further away, maybe it does some multiplication on the location of
the vertices on screen coordinates?

\par This is close. It has to do with a matrix multiplication on the
3d vertices of an object, altering how it looks from far away. The
projection matrix is part of the pipeline to get objects in 3d space
to screen space coordinates.

\par Note: The lack of a projection matrix is called an orthographic projection.

\subsection*{View Model Matrix}
\addcontentsline{toc}{subsection}{View Model Matrix}
\par This has something to do with the view frustum? Perhaps this is
the matrix that we perform operations on to change the camera
position and angle?

\subsection*{Depth Stencil Image}
\addcontentsline{toc}{subsection}{Depth Stencil Image}
\par What is the depth stencil image? I mean stencils are like
outlines of stuff so maybe depth stencil image draws the outlines of
objects in the depth frame buffer? Is the Z buffer a separate frame buffer?

\par So a depth stencil image is actually a combination of buffers
(both the depth and stencil buffers).

\subsection*{Stencil Buffer}
\addcontentsline{toc}{subsection}{Stencil Buffer}
\par Let's review what a stencil buffer is. It is true that a stencil
buffer can be used for outlines, but more generally\textemdash
stencil buffers can be thought of as masks. The concept of masking is
where you can think of separating layers. So you may 'cutout' a layer
by masking, so that you can see to the next layer.

\par Another way to think of this as a subtractive process. Here is
an excerpt from a conversation with Gemini 2.5 Pro:
\begin{verbatim}
Pass 1: Draw the Object (to create the "cutout" mask).
        Turn off writing to the color and depth buffers.
        Turn on writing to the stencil buffer.
        Set the stencil operation to: "If a pixel passes the depth test,
        write the value 1 into the stencil buffer."
        Draw your mesh.

        Result: The screen looks unchanged, but the stencil buffer now
        contains a perfect silhouette (a "mask") of your object, filled
        with the number 1. Everything else is 0.

    Pass 2: Draw the Outline (the "big yellow" version).
        Turn on writing to the color buffer (so we can see the yellow).
        Turn off writing to the depth buffer (this prevents the outline
        from messing up future depth tests).
        Set the stencil test to: "Only draw a pixel if the stencil buffer value
        at that location is NOT 1 (i.e., it's 0)."
        Draw a slightly larger version of your mesh, all in yellow.

    The "Cutout" Happens:

        The GPU starts to draw the big yellow mesh.
        Where the original object was (the "inner layer"), the stencil buffer
        has a 1. The test (stencil_value != 1) fails, and the yellow pixel is
        discarded. This is your "cutout"!

        On the larger edges of the yellow mesh (the "thickness" you mentioned), the pixel falls outside the original silhouette. The stencil buffer there is 0. The test (stencil_value == 0) passes, and the yellow pixel is drawn.

Final Result: You are left with only the yellow outline.
\end{verbatim}

\subsection*{Struct vs. Class}
\addcontentsline{toc}{subsection}{Struct vs. Class}
\par One thing that vkguide.dev has brought to my attention is when
to use struct versus when to use a class. From my understanding, a
struct is used for simple instances when we just need to store a
group of data. A class can hold lots of things for example state,
functionality, and data too.

\subsection*{Reading Code Snippet And Some Resulting Questions}
\addcontentsline{toc}{subsection}{Reading Code Snippet And Some
Resulting Questions}
\par When reading through Sascha's examples I came across this segment from
triangle.cpp:

\begin{verbatim}
// Create a view for the depth stencil image
// Images aren't directly accessed in Vulkan, but rather through views
// described by a subresource range This allows for multiple views of one
// image with differing ranges (e.g. for different layers)
VkImageViewCreateInfo depthStencilViewCI{};
depthStencilViewCI.sType = VK_STRUCTURE_TYPE_IMAGE_VIEW_CREATE_INFO;
depthStencilViewCI.viewType = VK_IMAGE_VIEW_TYPE_2D;
depthStencilViewCI.format = depthFormat;
depthStencilViewCI.subresourceRange = {};
depthStencilViewCI.subresourceRange.aspectMask = VK_IMAGE_ASPECT_DEPTH_BIT;
\end{verbatim}

\par What is a view? It seems that we first initialize the view and
then there are several instance variables we set for it. This code
snippet reveals many different questions:
\begin{itemize}
  \item What is a \textbf{subresource}?
  \item What is an \textbf{aspect}?
  \item What what does '\textbf{CI}' mean in 'depthStencilViewCI'?
\end{itemize}

\par I also have some more questions as a result from this snippet:
\begin{itemize}
  \item What kinds of things do we need a view for?
  \item Does Sascha abstract some of this 'state stuff' away when
    intializing a view?
  \item What does an all caps variable like
    'VK\_STRUCTURE\_TYPE\_IMAGE\_VIEW\_CREATE\_INFO'| mean again? Is it a
    global constant?
  \item \textemdash and finally, how much of this is important to me
    as an engine developer?
\end{itemize}

\subsection*{What Does 'CI' Suffix Mean In Variable Names?}
\addcontentsline{toc}{subsection}{What Does 'CI' Suffix Mean In Variable Names?}
\par CI is actually shorthand for \textbf{Create Info}. This is
because this variable is a \textbf{CreateInfo struct}, same as we
mentioned earlier.

\subsection*{VK View}
\addcontentsline{toc}{subsection}{VK View}
\par Images cannot be directly accessed in Vulkan, but instead with a
View object. Hence the two Vulkan objects:
\begin{itemize}
  \item VkImage
  \item VkImageView
\end{itemize}

\subsection*{How Does A View Relate To A Frame Buffer?}
\addcontentsline{toc}{subsection}{How Does A View Relate To A Frame Buffer?}
\par A frame buffer has views as attachments?

\par A \textbf{VkImage} is a collection of the image's raw data and
overall properties such as:
\begin{itemize}
  \item Dimension
  \item Format
  \item Mip Levels
\end{itemize}

\subsection*{Subresource}
\addcontentsline{toc}{subsection}{Subresource}
\par What is a subresource and how can it have a range?

\subsection*{VK\_NULL\_HANDLE}
\addcontentsline{toc}{subsection}{VK\_NULL\_HANDLE}
\par Weirdly, the VK\_NULL\_HANDLE is an integer, and not a null
pointer or something like that. We use VK\_NULL\_HANDLE| when we
are declaring some Vulkan object without giving it a value.

\subsection*{Ticket Number}
\addcontentsline{toc}{subsection}{Ticket Number}
What is a \textbf{ticket number}? What does it mean if there is an
\textbf{empty ticket number}? What is a \textbf{ticket slot}, and
what does it mean for it to be empty (0)?

\subsection*{How is initialization abstracted with Sascha's Framework?}
\addcontentsline{toc}{subsection}{How is initialization abstracted
with Sascha's Framework?}
\par The following is a snippet from a conversation with Gemini 2.5 Pro:
\begin{verbatim}
Before: Raw Vulkan (What we discussed)

You have to manually zero-initialize every struct and set every single
member, including the sType.

// 1. Define the binding
VkDescriptorSetLayoutBinding uboBinding{};
uboBinding.sType = ... // Easy to forget! Oh wait, this one doesn't have sType.
uboBinding.binding = 0;
uboBinding.descriptorType = VK_DESCRIPTOR_TYPE_UNIFORM_BUFFER;
uboBinding.descriptorCount = 1;
uboBinding.stageFlags = VK_SHADER_STAGE_VERTEX_BIT;
uboBinding.pImmutableSamplers = nullptr;

// 2. Define the create info
VkDescriptorSetLayoutCreateInfo layoutInfo{};
layoutInfo.sType = VK_STRUCTURE_TYPE_DESCRIPTOR_SET_LAYOUT_CREATE_INFO;
layoutInfo.pNext = nullptr;
layoutInfo.flags = 0;
layoutInfo.bindingCount = 1;
layoutInfo.pBindings = &uboBinding;

After: With Sascha Willems' vks::initializers

He provides helper functions that act as "constructors" for these C structs.
They take the important parameters and set all the other members (sType, pNext,
flags, etc.) to correct default values.

// 1. Define the binding (one line)
VkDescriptorSetLayoutBinding uboBinding =
    vks::initializers::descriptorSetLayoutBinding(
        VK_DESCRIPTOR_TYPE_UNIFORM_BUFFER,
        VK_SHADER_STAGE_VERTEX_BIT,
        0); // type, stage, binding

// 2. Define the create info (one line)
VkDescriptorSetLayoutCreateInfo layoutInfo =
    vks::initializers::descriptorSetLayoutCreateInfo(
        &uboBinding,
        1); // pBindings, bindingCount
\end{verbatim}
This example may prove useful when learning Sascha's framework.

\subsection*{What is multisampling (MSAA?)}
\addcontentsline{toc}{subsection}{What is multisampling (MSAA?)}
\par Multisampling (MSAA) is a method of anti aliasing. Aliasing happens
when a single sample inside of the pixel does not detect the
sub\-pixel fragment, so the result of the pixel can become an
artifact. Sub\-pixel artifacting are called \textbf{jaggies}, which
are a result of aliasing. To address this issue, we have multiple
sub\-pixel samples\textemdash hence the name.

\subsection*{Device vs. Physical Device}
\par In vulkan, there are two kinds of devices:
\begin{itemize}
  \item The \textbf{physical device (VkPhysicalDevice)}
  \item And the \textbf{logical device (VkDevice)}
\end{itemize}
\par Most always in the context of Vulkan, we are talking about the
logical device. The logical device is the object that you use to
interface with the hardware. You initialize VkDevice with the
tools/features that you are going to be using from the GPU.

\clearpage

\section*{C++ Notes}
\subsection*{Constructor Definition Syntax}
\par The following is a way to declare a new constructor:

\begin{verbatim}
VulkanExample() : VulkanExampleBase() {}
\end{verbatim}

\par More generally, it is written as:
\begin{verbatim}
constructorFromDerivedClass?() : NewConstructor?()
\end{verbatim}

\par I am very confused\textemdash but maybe it is unimportant

\subsection*{How To Make A Class}
\par In C++, to make a class we usually do the following:
\begin{enumerate}
  \item Declare the class in a header (e.g. class.h)
  \item Implement the class functionality in a source file (e.g. class.cpp)
\end{enumerate}

\subsection*{How To Make A Template Class?}
\par A template has to do with implementing functionality across
different primitive types? Like overloading function sum so that
sum(int a, int b) and sum(double a, double b)? What makes templates
different than classes? Do we use headers for them too?

\subsection*{}

\clearpage

\section*{Appendix}
\addcontentsline{toc}{section}{Appendix}
\begin{figure}[H]
  \begin{center}
    \includegraphics[width=0.95\textwidth]{figures/vulkanpipeline.png}
  \end{center}
  \caption{Vulkan Pipeline (Why is it so low res :())}\label{fig:}
\end{figure}

\begin{figure}[H]
  \begin{center}
    \includegraphics[width=0.95\textwidth]{figures/vulkanpipeline2.png}
  \end{center}
  \caption{Vulkan Pipeline (Why is it so low res :())}\label{fig:}
\end{figure}

\end{document}
